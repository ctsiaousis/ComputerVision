\documentclass[11pt]{scrartcl} % Font size

\input{structure.tex} % Include the file specifying the document structure and custom commands
\input{matlab.tex}

\usepackage{fontspec}
\setmainfont{Tinos Nerd Font} %nice font for english and greek

\begin{document}

\begin{titlepage}
    \centering
    \includegraphics[width=0.9\textwidth]{logo_tuc.png}\par\vspace{1cm}
    \normalfont\normalsize
    \textsc{\textcolor[rgb]{0.66, 0.09, 0.19}{Ηλεκτρολόγων Μηχανικών και Μηχανικών Υπολογιστών}}\\ % Your university, school and/or department name(s)
    \vspace{25pt} % Whitespace
    %\textcolor[rgb]{0.66, 0.09, 0.19}{\rule{\linewidth}{0.5pt}}
    \rule{\linewidth}{0.5pt}\\ % Thin top horizontal rule
    \vspace{20pt} % Whitespace
    {\Huge Μηχανική Όραση}\\ % The assignment title

    {\huge Αναφορά Τρίτου Project}\\ % The assignment title
    \vspace{12pt} % Whitespace
    \rule{\linewidth}{2pt}\\ % Thick bottom horizontal rule
    \vspace{12pt} % Whitespace
    \vspace{2cm}

    {\LARGE{Τσιαούσης Χρήστος \hfill Μητσάκης Χαράλαμπος}
        \par
        \texttt{2016030017 \hfill 2006030075}
        \par
    }

    \vfill
    Διδάσκουσα

    Ε. Δούτση

    \vfill

% Bottom of the page
    {\large \today\par}
\end{titlepage}

\newpage



\section{Εισαγωγή}

Η συγκεκριμένη εργασία διερευνά την αναγνώριση αντικειμένων σε μια εικόνα
βάσει της μεθοδολογίας Histogram of Oriented Gradients και Sliding Window Classification.

Αποτελείται από τα παρακάτω μέρη:
\begin{itemize}
  \item \textbf{Gradient Generation} Η συνάρτηση \texttt{mygradient} παράγει τις κλίσεις των
  magnitude και orientation για κάθε pixel της εικόνας εισόδου.
  \item \textbf{Histogram of Oriented Gradients} Η συνάρτηση \texttt{hog} παράγει τα orientation
  histograms για κάθε 8x8 superpixel της εικόνας εισόδου.
  \item \textbf{Template Detection} Η συνάρτηση \texttt{detect} δέχεται μία εικόνα εισόδου καθώς
  και κάποια template θετικών κι αρνητικών match. Επιστρέφει της συντεταγμένες και τα σκορ για
  τα καλύτερα matches που βρήκε.
\end{itemize}

\section{Αποτελέσματα}

\subsection{Image Gradient}

Καλώντας την mygradient ως εξής:

\begin{verbatim}
mygradient(im2double(rgb2gray(imread('../data/test3.jpg'))));
\end{verbatim}

παίρνουμε τα παρακάτω αποτελέσματα:

\begin{figure}[H]
  \includegraphics[width=16cm]{../output/1_mag.png}
  \caption{magnitude}
\end{figure}

\begin{figure}[H]
  \includegraphics[width=16cm]{../output/1_ori.png}
  \caption{orientation}
\end{figure}

\subsection{Detection}

\begin{figure}[H]
  \includegraphics[width=16cm,trim={0 2cm 0 2cm},clip]{../output/3_positive.png}
  \caption{positive template}
\end{figure}

\begin{figure}[H]
  \includegraphics[width=16cm,trim={0 2cm 0 2cm},clip]{../output/3_negative.png}
  \caption{negative template}
\end{figure}

\begin{figure}[H]
  \includegraphics[width=16cm,trim={0 4cm 0 4cm},clip]{../output/3_result_1.png}
  \caption{faces\_1b.jpg}
\end{figure}

\begin{figure}[H]
  \includegraphics[width=16cm,trim={3cm 1cm 3cm 1cm},clip]{../output/3_result_2.png}
  \caption{faces\_1b.jpg rotated $90^{\circ}$}
\end{figure}

\begin{figure}[H]
  \includegraphics[width=16cm,trim={0 3cm 0 3cm},clip]{../output/3_result_3.png}
  \caption{faces\_1b.jpg scaled x2}
\end{figure}

\begin{figure}[H]
  \includegraphics[width=16cm,trim={2cm 2cm 2cm 2cm},clip]{../output/3_result_4.png}
  \caption{faces2.jpg}
\end{figure}

\section{Συμπεράσματα}

Φαίνεται η μέθοδος να μην επιρρεάζεται από το orientation των αντικειμένων αλλά
δεν αναγνωρίζει απόλυτα templates διαφορετικού μεγέθους. Για να αντιμετοπιστεί
το πρόβλημα αυτό θα μπορούσαν να εφαρμοστούν ποικίλες υλοποιήσεις. Για παράδειγμα,
κάποιος θα μπορούσε να δημιουργήσει ένα \textbf{array από scaling factors} έτσι
ώστε να δημιουργηθούν διάφορα training sets, όπου τα επιλεγμένα feature του χρήστη
επιλέγονται και σε διαφορετικά scales με αποτέλεσμα να γίνει η αναγνώριση invariant
από το μέγεθος του εκάστοτε feature στο testing set. Μια άλλη ιδέα θα ήταν να χρησιμοποιήσουμε
διαφορετική μέθοδο classification. Αφού τα labels είναι μόνο δύο (match/όχι match)
θα μπορούσαμε να υλοποιήσουμε εκπαίδευση με κάποιο SupportVectorMachine για καλύτερο
διαχωρισμό των κλάσεων.



\end{document}
