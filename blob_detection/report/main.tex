\documentclass[11pt]{scrartcl} % Font size

\input{structure.tex} % Include the file specifying the document structure and custom commands
\input{matlab.tex}

\usepackage{fontspec}
\setmainfont{Tinos Nerd Font} %nice font for english and greek

\begin{document}

\begin{titlepage}
    \centering
    \includegraphics[width=0.9\textwidth]{logo_tuc.png}\par\vspace{1cm}
    \normalfont\normalsize
    \textsc{\textcolor[rgb]{0.66, 0.09, 0.19}{Ηλεκτρολόγων Μηχανικών και Μηχανικών Υπολογιστών}}\\ % Your university, school and/or department name(s)
    \vspace{25pt} % Whitespace
    %\textcolor[rgb]{0.66, 0.09, 0.19}{\rule{\linewidth}{0.5pt}}
    \rule{\linewidth}{0.5pt}\\ % Thin top horizontal rule
    \vspace{20pt} % Whitespace
    {\Huge Μηχανική Όραση}\\ % The assignment title

    {\huge Αναφορά Δεύτερου Project}\\ % The assignment title
    \vspace{12pt} % Whitespace
    \rule{\linewidth}{2pt}\\ % Thick bottom horizontal rule
    \vspace{12pt} % Whitespace
    \vspace{2cm}

    {\LARGE{Τσιαούσης Χρήστος \hfill Μητσάκης Χαράλαμπος}
        \par
        \texttt{2016030017 \hfill 2006030075}
        \par
    }

    \vfill
    Διδάσκουσα

    Ε. Δούτση

    \vfill

% Bottom of the page
    {\large \today\par}
\end{titlepage}

\newpage
% \tableofcontents


\section{Εισαγωγή}

\section{Υλοποίηση}

\subsection{Δημιουργία φίλτρων και scale-space}

Απο το paper του Lowe γνωρίζουμε ότι το scale-space μπορεί να υπολογιστεί με συνέλιξη της εικόνας με τη διαφορά δύο gaussian φίλτρων ως εξής:

\[D(x,y,\sigma) = (G(x,y,k\sigma) − G(x,y,\sigma)) * I(x,y)\]

Όμως η διαφορά δύο gaussian φίλτρων μπορεί να προσεγγυστεί απο το laplacian of gaussian ως εξής:

\[G(x,y,k\sigma) − G(x,y,\sigma) \approx (k−1)\sigma^2 \nabla^2 G\]

Άρα:

\[D(x,y,\sigma) = ((k−1)\sigma^2 \nabla^2 G) * I(x,y)\]

Επομένως δημιουργούμε φίλτρα με περιττές διαστάσεις ως εξής:

\begin{verbatim}
for i = 1:log_scales_per_octave
  k_power = i-1;
  sigma_prime = k^k_power * sigma;
  log_filters{i} = (k - 1) * sigma_prime^2 * ...
    fspecial('log', floor(4*sigma_prime) * 2 + 1, sigma_prime);
end
\end{verbatim}

Επιλέξαμε περιττές διαστάσεις ώστε να έχουμε κεντρικό pixel και αυτό να συμπίπτει με το κέντρο της laplacian of gaussian καμπάνας.

Αντί να δημιουργούμε φίλτρα για όλα τα scales χωρίζουμε το scale-space σε οκτάβες.

Η πρώτη εικόνα κάθε οκτάβας υπολογίζεται με υποδειγματοληψία της προηγούμενης οκτάβας στο $1/2$ του μεγέθους, εκτός απο την πρώτη οκτάβα που δημιουργείται με διπλασιασμό της αρχικής εικόνας με linear interpolation.

Κάθε οκτάβα αντιστοιχεί σε διπλασιασμό του $\sigma$, αλλα είναι πιο αποδοτικό, αντί να συνεχίζουμε να αυξάνουμε το $\sigma$, να κάνουμε downsample την εικόνα και να επαναχρησιμοποιούμε τα φίλτρα.

Έστω $s$ ο αριθμός των scales ανα οκτάβα στα οποία κάνουμε αναζήτηση extrema.
Κάθε οκτάβα περιέχει $s+2$ εικόνες. Γιαυτό δημιουργούμε $s+2$ φίλτρα τα οποία επαναχρησιμοποιούμε σε κάθε οκτάβα.
Οι 2 επιπλέον εικόνες υπάρχουν για να μπορέσουμε να κάνουμε αναζήτηση των extrema σε $s$ εικόνες γιατί για την αναζήτηση χρειαζόμαστε το επόμενο και το προηγούμενο scale όπως εξηγούμε παρακάτω (Εύρεση extrema).

Το $k$ υπολογίζεται απο το $s$ ως εξής: $k = 2^{1/s}$ γιατί κάθε οκτάβα αντιστοιχεί σε διπλασιασμό του $\sigma$ και η οκτάβα χωρίζεται σε $s$ διαστήματα.

Επιλέξαμε τις τιμές $\sigma = 1.6$, $k = 2^{1/3}$, $s = 3$ που είναι οι προτεινόμενες τιμές απο τον Lowe.
Επίσης επιλέξαμε $3$ οκτάβες, άρα έχουμε $n = 3 \cdot 5 = 15$ επίπεδα στα οποία κάναμε αναζήτηση extrema.

Το scale-space αποθηκεύεται σε ένα cell array με διαστάσεις $num\_of\_octaves \times (s+2)$ και κάθε στοιχείο του cell array περιέχει μια εικόνα.
\begin{verbatim}
scale_space = cell(num_of_octaves, s+2);
\end{verbatim}

\subsection{Εύρεση extrema}

Για να βρούμε τα τοπικά maxima και minima του scale-space (υποψήφια keypoints) συγκρίνουμε κάθε σημείο του scale-space με τα 26 γειτονικά του (8 απο το ίδιο scale + 9 απο το κατώτερο + 9 απο το ανώτερο scale).

Απορρίπτουμε υποψήφια keypoints με τιμές κοντά στο 0 κάτω απο ένα threshold.

Επίσης απορρίπτουμε υποψήφια keypoints στα edges ελέγχοντας αν ισχύει η σχέση (για $r = 10$):
\[\frac{Tr(H)^2}{Det(H)} < \frac{(r+1)^2}{r}\]

\subsection{Απεικόνιση blobs}

Όταν βρούμε ένα keypoint αποθηκεύουμε τις συντεταγμένες στις οποίες το βρήκαμε πολλαπλασιασμένες ανάλογα την οκτάβα ($ο$) στην οποία βρισκόμαστε:
\[x' = x \cdot \frac{2^{o-1}}{2}\]
\[y' = y \cdot \frac{2^{o-1}}{2}\]

Πολλαπλασιάζουμε με $\frac{2^{o-1}}{2}$ γιατί σε κάθε οκτάβα οι διαστάσεις της εικόνας διαιρούνται με 2 αλλα στην πρώτη οκτάβα πολλαπλασιάζονται με 2.

Επίσης αποθηκεύουμε την ακτίνα η οποία είναι το μέγεθος του φίλτρου στο scale ($i$) που βρισκόμαστε $k^{i-1} \sigma$ πολλαπλασιασμένο με $\frac{2^{o-1}}{2}$:
\[r = k^{i-1} \sigma \cdot \frac{2^{o-1}}{2}\]

\begin{verbatim}
rowVector = [rowVector; y .* (2^(o)/4)];
colVector = [colVector; x .* (2^(o)/4)];
radiusVector = [radiusVector; k^(sc-1) * sigma * (2^(o)/4)];
\end{verbatim}

\section{Αποτελέσματα}

\begin{figure}[H]
  \centerline{\includegraphics[width=24cm,trim={0 5cm 0 5cm},clip]{../output/fishes.jpg}}
  \caption{}
\end{figure}

\begin{figure}[H]
  \centerline{\includegraphics[width=24cm,trim={0 3cm 0 3cm},clip]{../output/sunflowers.jpg}}
  \caption{}
\end{figure}

\begin{figure}[H]
  \centerline{\includegraphics[width=24cm,trim={0 6cm 0 6cm},clip]{../output/yacht.jpg}}
  \caption{}
\end{figure}

\begin{figure}[H]
  \centerline{\includegraphics[width=24cm,trim={0 5cm 0 5cm},clip]{../output/otter.jpg}}
  \caption{}
\end{figure}

\section{Συμπεράσματα}

\end{document}
