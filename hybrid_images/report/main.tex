\documentclass[11pt]{scrartcl} % Font size

\input{structure.tex} % Include the file specifying the document structure and custom commands
\input{matlab.tex}

\usepackage{fontspec}
\setmainfont{Tinos Nerd Font} %nice font for english and greek


\begin{document}

\begin{titlepage}
    \centering
    \includegraphics[width=0.9\textwidth]{logo_tuc.png}\par\vspace{1cm}
    \normalfont\normalsize
    \textsc{\textcolor[rgb]{0.66, 0.09, 0.19}{Ηλεκτρολόγων Μηχανικών και Μηχανικών Υπολογιστών}}\\ % Your university, school and/or department name(s)
    \vspace{25pt} % Whitespace
    %\textcolor[rgb]{0.66, 0.09, 0.19}{\rule{\linewidth}{0.5pt}}
    \rule{\linewidth}{0.5pt}\\ % Thin top horizontal rule
    \vspace{20pt} % Whitespace
    {\Huge Μηχανική Όραση}\\ % The assignment title

    {\huge Αναφορά Πρώτου Project}\\ % The assignment title
    \vspace{12pt} % Whitespace
    \rule{\linewidth}{2pt}\\ % Thick bottom horizontal rule
    \vspace{12pt} % Whitespace
    \vspace{2cm}

    {\LARGE{Τσιαούσης Χρήστος \hfill Μητσάκης Χαράλαμπος}
        \par
        \texttt{2016030017 \hfill 2006030075}
        \par
    }

    \vfill
    Διδάσκων

    Ε. Δούτση

    \vfill

% Bottom of the page
    {\large \today\par}
\end{titlepage}

\newpage
\tableofcontents


\section{Εισαγωγή}

\section{Υλοποίηση}

\subsection{Συνάρτηση my\_imfilter}

%\matlabscript {file}{caption} %κληση του αρχείου χωρίς το .m

\subsection{Hybrid Images}

\subsubsection{Συχνότητα αποκοπής}

Ως συχνότητα αποκοπής $f_c$ θεωρούμε τη συχνότητα όπου το φίλτρο μειώνει το amplitude του σήματος στο μισό.

Η συχνότητα αποκοπής για τιμή αποκοπής $1/c$ σχετίζεται με την τυπική απόκλιση στο πεδίο των συχνοτήτων με την εξής σχέση:
\[f_c = \sqrt{2\ln(c)}\cdot\sigma_f\]

Επειδή θέλουμε τιμή αποκοπής $1/2$ θα είναι $c = 2$.

Λύνοντας ως προς $\sigma_f$ έχουμε:
\[\sigma_f = \frac{f_c}{\sqrt{2\ln(2)}}\]

Στη συνέχεια υπολογίζουμε την τυπική απόκλιση στο πεδίο του χώρου ($\sigma$) απο τη σχέση:
\[\sigma\cdot\sigma_f=\frac{N}{2\pi}\]

Η τυπική απόκλιση ($\sigma$) χρησιμοποιείται για τη δημιουργία του Gaussian φίλτρου.

\section{Αποτελέσματα}

% \listoffigures
\end{document}
