\documentclass[11pt]{scrartcl} % Font size

\input{structure.tex} % Include the file specifying the document structure and custom commands
\input{matlab.tex}

\usepackage{fontspec}
\setmainfont{Tinos Nerd Font} %nice font for english and greek

\begin{document}

\begin{titlepage}
    \centering
    \includegraphics[width=0.9\textwidth]{logo_tuc.png}\par\vspace{1cm}
    \normalfont\normalsize
    \textsc{\textcolor[rgb]{0.66, 0.09, 0.19}{Ηλεκτρολόγων Μηχανικών και Μηχανικών Υπολογιστών}}\\ % Your university, school and/or department name(s)
    \vspace{25pt} % Whitespace
    %\textcolor[rgb]{0.66, 0.09, 0.19}{\rule{\linewidth}{0.5pt}}
    \rule{\linewidth}{0.5pt}\\ % Thin top horizontal rule
    \vspace{20pt} % Whitespace
    {\Huge Μηχανική Όραση}\\ % The assignment title

    {\huge Αναφορά Πρώτου Project}\\ % The assignment title
    \vspace{12pt} % Whitespace
    \rule{\linewidth}{2pt}\\ % Thick bottom horizontal rule
    \vspace{12pt} % Whitespace
    \vspace{2cm}

    {\LARGE{Τσιαούσης Χρήστος \hfill Μητσάκης Χαράλαμπος}
        \par
        \texttt{2016030017 \hfill 2006030075}
        \par
    }

    \vfill
    Διδάσκουσα

    Ε. Δούτση

    \vfill

% Bottom of the page
    {\large \today\par}
\end{titlepage}

\newpage
% \tableofcontents


\section{Εισαγωγή}

Ο σκοπός του project είναι η δημιουργία υβριδικών εικόνων που προκύπτουν απο τη συνένωση δύο εικόνων με τέτοιο τρόπο ώστε το αποτέλεσμα να μοιάζει με τη μία εικόνα όταν το βλέπουμε απο κοντά και με την άλλη όταν το βλέπουμε απο μακριά.

\section{Υλοποίηση}

\subsection{Συνάρτηση my\_imfilter}

Αρχικά προσθέτουμε στην εικόνα symmetric padding για να έχουμε καλύτερο αποτέλεσμα στα άκρα της εικόνας. Υλοποιήσαμε το padding χωρις χρήση της συνάρτησης $padarray()$.

Υλοποιήσαμε το φιλτράρισμα ως συνέλιξη στο χωρικό πεδίο. Η συνέλιξη ισοδυναμεί με συσχέτιση της εικόνας με το περιστραμμένο κατα $180^{\circ}$ φίλτρο.

Η συνέλιξη πραγματοποιήται για κάθε κανάλι της εικόνας ξεχωριστά. Έχουμε 3 εμφωλευμένα loops για τα κανάλια, τις γραμμές και τις στήλες της εικόνας.

% \matlabscript {../src/my_imfilter}{caption} %κληση του αρχείου χωρίς το .m

\subsection{Υβριδικές εικόνες}

\subsubsection{Συχνότητα αποκοπής}

Ως συχνότητα αποκοπής $f_c$ θεωρούμε τη συχνότητα όπου το gaussian φίλτρο μειώνει το amplitude του σήματος στο μισό.

Η συχνότητα αποκοπής για τιμή αποκοπής $1/c$ σχετίζεται με την τυπική απόκλιση στο πεδίο των συχνοτήτων με την εξής σχέση \cite{wikipedia_gaussian_filter}:
\[f_c = \sqrt{2\ln(c)}\cdot\sigma_f\]

Επειδή θέλουμε τιμή αποκοπής $1/2$ θα είναι $c = 2$.

Λύνοντας ως προς $\sigma_f$ έχουμε:
\[\sigma_f = \frac{f_c}{\sqrt{2\ln(2)}}\]

Στη συνέχεια υπολογίζουμε την τυπική απόκλιση στο πεδίο του χώρου ($\sigma$) απο τη σχέση:
\[\sigma\cdot\sigma_f=\frac{N}{2\pi}\]

Η τυπική απόκλιση ($\sigma$) χρησιμοποιείται για τη δημιουργία του gaussian φίλτρου.

\subsubsection{Φίλτρα}

Χρησιμοποιούμε gaussian φίλτρα με $\sigma$ την τιμή που υπολογίσαμε απο το $f_c$ όπως εξηγήσαμε παραπάνω.

Η υβριδική εικόνα προκύπτει απο το άθροισμα δυο φιλτραρισμένων εικόνων. Η μία προκύπτει απο την εφαρμογή του gaussian φίλτρου στην πρώτη εικόνα με αποτέλεσμα να κρατάμε τις χαμηλές συχνότητες, και η άλλη προκύπτει απο την εφαρμοφή του gaussian φίλτρου στην δεύτερη εικόνα και ύστερα αφαίρεση του αποτελέσματος απο την εικόνα με αποτέλεσμα να κρατάμε τις υψηλές συχνότητες.

Χρησιμοποιούμε διαφορετική συχνότητα αποκοπής για το low-pass και το high-pass φίλτρο ώστε να υπάρχει ένα κενό στις συχνότητες και να είναι πιο ευδιάκριτες οι δύο εικόνες στο υβριδικό αποτέλεσμα. Απο μια συχνότητα αποκοπής $f_c$ υπολογίζουμε δύο άλλες: $f_c \cdot 0.8$ και $f_c \cdot 1.2$ απο τις οποίες παίρνουμε δύο διαφορετικά $\sigma$.

% \matlabscript {../src/assignment1}{caption}

\section{Αποτελέσματα}

\subsection{Einstein - Marilyn}

Ύστερα απο πειραματισμό επιλέξαμε συχνότητα αποκοπής $f_c = 19$ η οποία μας δίνει $\sigma = 3.26$ για το low pass και $\sigma = 2.17$ για το high pass φίλτρο.

Σε αντίθεση με τα άλλα ζευγάρια εικόνων στο συγκεκριμένο δεν είναι προφανές ποιά εικόνα είναι πιο αναγνωρίσιμη στις υψηλές και ποιά στις χαμηλές συχνότητες οπότε δοκιμάσατε και τις δύο.

Αρχικά εφαρμόσαμε το low pass φίλτρο στην εικόνα της Marilyn και το high pass φίλτρο στην εικόνα του Einstein.

\begin{figure}[H]
  \begin{minipage}[c]{8cm}
    \includegraphics[width=7cm]{../output/pair1_marilyn.bmp__pair1_einstein.bmp__low_frequencies__fc19_g20.jpg}
    \caption{Marilyn - χαμηλές συχνότητες}
  \end{minipage}
  \begin{minipage}[c]{8cm}
    \includegraphics[width=7cm]{../output/pair1_marilyn.bmp__pair1_einstein.bmp__high_frequencies__fc19_g20.jpg}
    \caption{Einstein - υψηλές συχνότητες}
  \end{minipage}
\end{figure}

\begin{figure}[H]
  \includegraphics[width=16cm]{../output/pair1_marilyn.bmp__pair1_einstein.bmp__hybrid_image_scales__fc19_g20.jpg}
  \caption{υβριδική εικόνα}
\end{figure}

Στις εικόνες με μεγάλο μέγεθος βλέπουμε ξακάθαρα τον Einstein. Όσο μικραίνει η εικόνα χάνονται οι λεπτομέριες τους προσώπου του Einstein και κυριαρχεί η εικόνα της Marilyn.

Στη συνέχεια εφαρμόσαμε το low pass φίλτρο στην εικόνα του Einstein και το high pass φίλτρο στην εικόνα της Marilyn.

\begin{figure}[H]
  \begin{minipage}[c]{8cm}
    \includegraphics[width=7cm]{../output/pair1_einstein.bmp__pair1_marilyn.bmp__low_frequencies__fc19_g20.jpg}
    \caption{Marilyn - χαμηλές συχνότητες}
  \end{minipage}
  \begin{minipage}[c]{8cm}
    \includegraphics[width=7cm]{../output/pair1_einstein.bmp__pair1_marilyn.bmp__high_frequencies__fc19_g20.jpg}
    \caption{Einstein - υψηλές συχνότητες}
  \end{minipage}
\end{figure}

\begin{figure}[H]
  \includegraphics[width=16cm]{../output/pair1_einstein.bmp__pair1_marilyn.bmp__hybrid_image_scales__fc19_g20.jpg}
  \caption{υβριδική εικόνα}
\end{figure}

Στις εικόνες με μεγάλο μέγεθος βλέπουμε ξακάθαρα την Marilyn. Όσο μικραίνει η εικόνα χάνονται οι λεπτομέριες τους προσώπου της Marilyn και κυριαρχεί η εικόνα του Einstein.

\subsection{Joker - Heath Ledger}

Ύστερα απο πειραματισμό επιλέξαμε συχνότητα αποκοπής $f_c = 11$ η οποία μας δίνει $\sigma = 13.58$ για το low pass και $\sigma = 9.05$ για το high pass φίλτρο.

Εφαρμόσαμε το low pass φίλτρο στην εικόνα του Joker και το high pass φίλτρο στην εικόνα του Heath Ledger. Παρόλο που η εικόνα του Joker έχει περισσότερες υψηλές συχνότητες, η φιγούρα του Joker είναι πιο αναγνωρίσιμη απο τη βαφή του παρά απο τη φυσιογνωμία του προσώπου του. Γιαυτό είναι πιο αναγωρίσιμη στις χαμηλές συχνότητες.
\begin{figure}[H]
  \begin{minipage}[c]{9cm}
    \includegraphics[width=8cm]{../output/pair2_joker.png__pair2_HeathLedger.png__low_frequencies__fc11_g20.jpg}
    \caption{Joker - χαμηλές συχνότητες}
  \end{minipage}
  \begin{minipage}[c]{9cm}
    \includegraphics[width=8cm]{../output/pair2_joker.png__pair2_HeathLedger.png__high_frequencies__fc11_g20.jpg}
    \caption{Heath Ledger - υψηλές συχνότητες}
  \end{minipage}
\end{figure}
\begin{figure}[H]
  \includegraphics[width=17cm]{../output/pair2_joker.png__pair2_HeathLedger.png__hybrid_image_scales__fc11_g20.jpg}
  \caption{υβριδική εικόνα}
\end{figure}
Στις εικόνες με μεγάλο μέγεθος βλέπουμε ξακάθαρα τον Heath Ledger ο οποίος φαίνεται σα να είναι βαμμένος στα χρώματα του Joker. Όσο μικραίνει η εικόνα χάνονται οι λεπτομέριες του προσώπου του Heath Ledger και κυριαρχεί η εικόνα του Joker.

\subsection{Πυραμίδα του Χέοπα - El Castillo}

Πέραν των εικόνων που μας δόθηκαν δοκιμάσαμε την ίδια τεχνική στην πυραμίδα του Χέοπα \cite{wikipedia_kheops} και το El Castillo \cite{britannica_el_castillo}. Επιλέξαμε αυτά γιατί έχουν παρόμοιο γεωμετρικό σχήμα που σημαίνει οτι θα υπάρχει καλό alignment των εικόνων.

Ύστερα απο πειραματισμό επιλέξαμε συχνότητα αποκοπής $f_c = 15$ η οποία μας δίνει $\sigma = 5.57$ για το low pass και $\sigma = 3.71$ για το high pass φίλτρο.

Εφαρμόσαμε το low pass φίλτρο στην πυραμίδα του Χέοπα και το high pass φίλτρο στο El Castillo γιατί η πυραμίδα του Χέοπα είναι αναγνωρίσιμη απο το χρώμα και το γεωμετρικό σχήμα της πυραμίδας χωρίς να μας ενδιαφέρουν οι λεπτομέριες, ενώ το El Castillo είναι αναγνωρίσιμο απο τις βαθμίδες και τις σκάλες που φαίνονται στις υψηλές συχνότητες.
\begin{figure}[H]
  \begin{minipage}[c]{9cm}
    \includegraphics[width=8cm]{../output/pair3_kheops_pyramid.jpg__pair3_mayan_pyramid.jpg__low_frequencies__fc15_g20.jpg}
    \caption{πυραμίδα Χέοπα - χαμηλές συχνότητες}
  \end{minipage}
  \begin{minipage}[c]{9cm}
    \includegraphics[width=8cm]{../output/pair3_kheops_pyramid.jpg__pair3_mayan_pyramid.jpg__high_frequencies__fc15_g20.jpg}
    \caption{El Castillo - υψηλές συχνότητες}
  \end{minipage}
\end{figure}
\begin{figure}[H]
  \includegraphics[width=17cm]{../output/pair3_kheops_pyramid.jpg__pair3_mayan_pyramid.jpg__hybrid_image_scales__fc15_g20.jpg}
  \caption{υβριδική εικόνα}
\end{figure}
Στις εικόνες με μεγάλο μέγεθος βλέπουμε ξακάθαρα το El Castillo σε πορτοκαλί απόχρωση καθώς παίρνει το χρώμα της πυραμίδας του Χέοπα. Όσο μικραίνει η εικόνα χάνονται οι λεπτομέριες του El Castillo και κυριαρχεί η πυραμίδα του Χέοπα.

Το αποτέλεσμα είναι πετυχημένο γιατί υπάρχει καλό alignment γιατί είναι τραβηγμένες απο ίδια γωνία οπότε συμπίπτουν ακόμα και οι ακμές των δύο πυραμίδων. Έτσι το χρώμα που παίρνει το El Castillo απο την πυραμίδα του Χέοπα φαίνεται φυσικό.

\section{Συμπεράσματα}

Η τεχνική που περιγράψαμε ενώνει δύο εικόνες σε μια με τέτοιο τρόπο ώστε το αποτέλεσμα όταν το βλέπουμε απο κοντά να μοιάζει με την εικόνα στην οποία εφαρμόσαμε το high-pass φίλτρο, και όταν το βλέπουμε απο μακριά να μοιάζει με την εικόνα στην οποία εφαρμόσατε το low-pass φίλτρο.

Παρατηρούμε οτι είναι καλύτερο να εφαρμόσουμε το low-pass φίλτρο στην εικόνα που είναι πιο αναγνωρίσιμη απο το σχήμα των αντικειμένων όπως αυτό φαίνεται απο μακριά,
και το high-pass φίλτρο στην εικόνα που είναι πιο αναγνωρίσιμη απο λεπτομέριες όπως μάτια ή σκάλες.

Επίσης είναι σημαντικό να υπάρχει καλό alignment.
%Έτσι μειώνεται η επίδραση της μιας εικόνας στην άλλη.
Όταν προσπαθούμε να διακρίνουμε στην υβριδική εικόνα την εικόνα με τις υψηλές συχνότητες, η εικόνα με τις χαμηλές συχνότητες μπορεί να μας αποσπάει την προσοχή ειδικά αν έχει συμμετρία ή επανάληψη ενός pattern.
Το alignment βοηθάει το μάτι να ερμηνεύσει την εικόνα με τις χαμηλές συχνότητες σαν χαρακτηριστικό (π.χ. σκιά, βαφή προσώπου) της εικόνα με υψηλές συχνότητες.

% ίσως και σχόλια για τα cutoff

% \listoffigures
\begin{thebibliography}{3}

\bibitem{wikipedia_gaussian_filter}
Wikipedia - Gaussian Filter,
\url{https://en.wikipedia.org/w/index.php?title=Gaussian_filter&oldid=1016628269#Digital_implementation}

\bibitem{wikipedia_kheops}
Wikipedia - Great Pyramid of Giza,
\url{https://en.wikipedia.org/w/index.php?title=Great_Pyramid_of_Giza&oldid=1018089268}

\bibitem{britannica_el_castillo}
Britannica - El Castillo,
\url{https://www.britannica.com/place/El-Castillo}

\end{thebibliography}

\end{document}
