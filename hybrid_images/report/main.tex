\documentclass[11pt]{scrartcl} % Font size

\input{structure.tex} % Include the file specifying the document structure and custom commands
\input{matlab.tex}

\usepackage{fontspec}
\setmainfont{Tinos Nerd Font} %nice font for english and greek


\begin{document}

\begin{titlepage}
    \centering
    \includegraphics[width=0.9\textwidth]{logo_tuc.png}\par\vspace{1cm}
    \normalfont\normalsize
    \textsc{\textcolor[rgb]{0.66, 0.09, 0.19}{Ηλεκτρολόγων Μηχανικών και Μηχανικών Υπολογιστών}}\\ % Your university, school and/or department name(s)
    \vspace{25pt} % Whitespace
    %\textcolor[rgb]{0.66, 0.09, 0.19}{\rule{\linewidth}{0.5pt}}
    \rule{\linewidth}{0.5pt}\\ % Thin top horizontal rule
    \vspace{20pt} % Whitespace
    {\Huge Μηχανική Όραση}\\ % The assignment title

    {\huge Αναφορά Πρώτου Project}\\ % The assignment title
    \vspace{12pt} % Whitespace
    \rule{\linewidth}{2pt}\\ % Thick bottom horizontal rule
    \vspace{12pt} % Whitespace
    \vspace{2cm}

    {\LARGE{Τσιαούσης Χρήστος \hfill Μητσάκης Χαράλαμπος}
        \par
        \texttt{2016030017 \hfill 2006030075}
        \par
    }

    \vfill
    Διδάσκων

    Ε. Δούτση

    \vfill

% Bottom of the page
    {\large \today\par}
\end{titlepage}

\newpage
\tableofcontents


\section{Εισαγωγή}

Ο σκοπός του project είναι η δημιουργία υβριδικών εικόνων που προκύπτουν απο τη συνένωση δύο εικόνων με τέτοιο τρόπο ώστε το αποτέλεσμα να μοιάζει με τη μία εικόνα όταν το βλέπουμε απο κοντά και με την άλλη όταν το βλέπουμε απο μακριά.

\section{Υλοποίηση}

\subsection{Συνάρτηση my\_imfilter}

%\matlabscript {file}{caption} %κληση του αρχείου χωρίς το .m

\subsection{Hybrid Images}

\subsubsection{Συχνότητα αποκοπής}

Ως συχνότητα αποκοπής $f_c$ θεωρούμε τη συχνότητα όπου το Gaussian φίλτρο μειώνει το amplitude του σήματος στο μισό.

Η συχνότητα αποκοπής για τιμή αποκοπής $1/c$ σχετίζεται με την τυπική απόκλιση στο πεδίο των συχνοτήτων με την εξής σχέση:
\[f_c = \sqrt{2\ln(c)}\cdot\sigma_f\]

Επειδή θέλουμε τιμή αποκοπής $1/2$ θα είναι $c = 2$.

Λύνοντας ως προς $\sigma_f$ έχουμε:
\[\sigma_f = \frac{f_c}{\sqrt{2\ln(2)}}\]

Στη συνέχεια υπολογίζουμε την τυπική απόκλιση στο πεδίο του χώρου ($\sigma$) απο τη σχέση:
\[\sigma\cdot\sigma_f=\frac{N}{2\pi}\]

Η τυπική απόκλιση ($\sigma$) χρησιμοποιείται για τη δημιουργία του Gaussian φίλτρου.

\section{Αποτελέσματα}

\subsection{Einstein - Marilyn}

Ύστερα απο πειραματισμό επιλέξαμε συχνότητα αποκοπής $f_c = 17$ η οποία μας δίνει $\sigma = 3.65$ για το low pass και $\sigma = 2.43$ για το high pass φίλτρο.

[εικόνες fourier?]
\begin{figure}[H]
  \begin{minipage}[c]{9cm}
    \includegraphics[width=8cm]{../output/pair1_marilyn.bmp__pair1_einstein.bmp__low_frequencies__fc17_g20.jpg}
    \caption{χαμηλές συχνότητες}
  \end{minipage}
  \begin{minipage}[c]{9cm}
    \includegraphics[width=8cm]{../output/pair1_marilyn.bmp__pair1_einstein.bmp__high_frequencies__fc17_g20.jpg}
    \caption{υψηλές συχνότητες}
  \end{minipage}
\end{figure}

Εφαρμόσαμε το low pass φίλτρο στην εικόνα της Marilyn και το high pass φίλτρο στην εικόνα του Einstein γιατί όπως φαίνεται απο την απεικόνιση του φάσματος η εικόνα του Einstein έχει περισσότερες υψηλές συχνότητες.

\begin{figure}[H]
  \includegraphics[width=17cm]{../output/pair1_marilyn.bmp__pair1_einstein.bmp__hybrid_image_scales__fc17_g20.jpg}
  \caption{}
\end{figure}

Στις εικόνες με μεγάλο μέγεθος βλέπουμε ξακάθαρα τον Einstein. Όσο μικραίνει η εικόνα χάνονται οι λεπτομέριες τους προσώπου του Einstein και κυριαρχεί η εικόνα της Marilyn.

\subsection{Joker - Heath Ledger}

Ύστερα απο πειραματισμό επιλέξαμε συχνότητα αποκοπής $f_c = 11$ η οποία μας δίνει $\sigma = 13.58$ για το low pass και $\sigma = 9.05$ για το high pass φίλτρο.

[εικόνες fourier]
[εικόνες low high frequencies]
\begin{figure}[H]
  \begin{minipage}[c]{9cm}
    \includegraphics[width=8cm]{../output/pair2_joker.png__pair2_HeathLedger.png__low_frequencies__fc11_g20.jpg}
    \caption{χαμηλές συχνότητες}
  \end{minipage}
  \begin{minipage}[c]{9cm}
    \includegraphics[width=8cm]{../output/pair2_joker.png__pair2_HeathLedger.png__high_frequencies__fc11_g20.jpg}
    \caption{υψηλές συχνότητες}
  \end{minipage}
\end{figure}

Εφαρμόσαμε το low pass φίλτρο στην εικόνα του Joker και το high pass φίλτρο στην εικόνα του Heath Ledger. Παρόλο που η εικόνα του Joker έχει περισσότερες υψηλές συχνότητες, η φιγούρα του Joker είναι πιο αναγνωρίσιμη απο τη βαφή του παρά απο τη φυσιογνωμία του προσώπου του. Γιαυτό είναι πιο αναγωρίσιμη στις χαμηλές συχνότητες.

\begin{figure}[H]
  \includegraphics[width=17cm]{../output/pair2_joker.png__pair2_HeathLedger.png__hybrid_image_scales__fc11_g20.jpg}
  \caption{}
\end{figure}

Στις εικόνες με μεγάλο μέγεθος βλέπουμε ξακάθαρα τον Heath Ledger ο οποίος φαίνεται σα να είναι βαμμένος στα χρώματα του Joker. Όσο μικραίνει η εικόνα χάνονται οι λεπτομέριες του προσώπου του Heath Ledger και κυριαρχεί η εικόνα του Joker.

\subsection{Πυραμίδα του Χέοπα - El Castillo}

Ύστερα απο πειραματισμό επιλέξαμε συχνότητα αποκοπής $f_c = 15$ η οποία μας δίνει $\sigma = 5.57$ για το low pass και $\sigma = 3.71$ για το high pass φίλτρο.

\begin{figure}[H]
  \begin{minipage}[c]{9cm}
    \includegraphics[width=8cm]{../output/pair3_kheops_pyramid.jpg__pair3_mayan_pyramid.jpg__low_frequencies__fc15_g20.jpg}
    \caption{χαμηλές συχνότητες}
  \end{minipage}
  \begin{minipage}[c]{9cm}
    \includegraphics[width=8cm]{../output/pair3_kheops_pyramid.jpg__pair3_mayan_pyramid.jpg__high_frequencies__fc15_g20.jpg}
    \caption{υψηλές συχνότητες}
  \end{minipage}
\end{figure}

Εφαρμόσαμε το low pass φίλτρο στην πυραμίδα του Χέοπα και το high pass φίλτρο στο El Castillo γιατί η πυραμίδα του Χέοπα είναι αναγνωρίσιμη απο το χρώμα και το γεωμετρικό σχήμα της πυραμίδας χωρίς να μας ενδιαφέρουν οι λεπτομέριες, ενώ το El Castillo είναι αναγνωρίσιμο απο τις βαθμίδες και τις σκάλες που φαίνονται στις υψηλές συχνότητες.

\begin{figure}[H]
  \includegraphics[width=17cm]{../output/pair3_kheops_pyramid.jpg__pair3_mayan_pyramid.jpg__hybrid_image_scales__fc15_g20.jpg}
  \caption{}
\end{figure}

Στις εικόνες με μεγάλο μέγεθος βλέπουμε ξακάθαρα το El Castillo σε πορτοκαλί απόχρωση καθώς παίρνει το χρώμα της πυραμίδας του Χέοπα. Όσο μικραίνει η εικόνα χάνονται οι λεπτομέριες του El Castillo και κυριαρχεί η πυραμίδα του Χέοπα.

Το αποτέλεσμα είναι πετυχημένο γιατί υπάρχει καλό alignment γιατί είναι τραβηγμένες απο ίδια γωνία οπότε συμπίπτουν ακόμα και οι ακμές των δύο πυραμίδων. Έτσι το χρώμα που παίρνει το El Castillo απο την πυραμίδα του Χέοπα φαίνεται φυσικό.

\subsection{Συμπεράσματα}

Παρατηρούμε οτι είναι καλύτερο να εφαρμόσουμε το low pass φίλτρο στην εικόνα που είναι πιο αναγνωρίσιμη απο το σχήμα των αντικειμένων όπως αυτό φαίνεται απο μακριά,
και το high pass φίλτρο στην εικόνα που είναι πιο αναγνωρίσιμη απο λεπτομέριες όπως μάτια ή σκάλες.

Επίσης είναι σημαντικό να υπάρχει καλό alignment.
%Έτσι μειώνεται η επίδραση της μιας εικόνας στην άλλη.
Όταν προσπαθούμε να διακρίνουμε στην υβριδική εικόνα την εικόνα με τις υψηλές συχνότητες, η εικόνα με τις χαμηλές συχνότητες μπορεί να μας αποσπάει την προσοχή ειδικά αν έχει συμμετρία ή επανάληψη ενός pattern.
Το alignment βοηθάει το μάτι να ερμηνεύσει την εικόνα με τις χαμηλές συχνότητες σαν χαρακτηριστικό (π.χ. σκιά, βαφή προσώπου) της εικόνα με υψηλές συχνότητες.

% ίσως και σχόλια για τα cutoff

% \listoffigures
\end{document}
